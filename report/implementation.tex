\chapter{Implementation}

\section{Data Model} % (fold)

\subsection{Database Schema} % (fold)
\label{sub:database_schema}

	In this project, we used three table for storing all the information from users. As Figure~\ref{fig:data-schema} shows, Events is the main table in the app. It provides fields ``address'', ``comment'', ``creationDate'', ``latitude'', ``longitude'', ``locationName'', ``rate'', ``thumbnail'', ``photoBlob'' and ``tags''. ``address'' field is used when user didn't find their ``locationName'' in location List fetching from Foursquare API v2. ``Latitude'' and ``longitude'' is used for adding annotations in Map View. A 80*80 resolution thumbnail is stored for each event in order to accelerate loading in food list table. \\ 
	
	``photoBlob'' has one to one relationship with ``photo'' in PhotoBlob table. Using a separate table should also help speeding up when we don't need to load photo while we still need to get the meta data of the event. \\
	
	Field ``tags'' has many to many relationship with ``photos'' in Tag table since one photo can labeled with many tags and one tag can relate to many photos.  
	
\label{sec:data_model}
\begin{figure}
	\centering
    \SetFigLayout{1}{1}
    {\includegraphics[%
    width=\figwidth, totalheight=\figheight, keepaspectratio]{./screenshots/database_schema.png}}
    \caption{Database Schema}
	\label{fig:data-schema}
\end{figure}

% subsection database_schema (end)

\subsection{Settings Property List} % (fold)
\label{sub:settings_list}

	When developing iOS, we can also use another way to store data we need. Property List is used to store all information related to the app itself. For instance, using Property List to store App Display Name, App Version are commonly used in all apps for iPhone. \\
	
	``Fancy Foodie'' uses release number as main version string and git hash tag of the release as build string, so ``1.0 (build 98a9e84)'' is shown in Figure~\ref{fig:settings} which means that the main version number is 1.0 and build hash tag is 98a9e84. \\
	 
	 This app also use property list to store whether we need to save photo to album locally. If this option is enabled, the app will create a album  named ``Fancy Foodie Photos'' and put photos in this album as shown in Figure~\ref{fig:album}.
	 
 	 
% subsection settings_list (end)
% section data_model (end)
\newpage

\section{Workflow} % (fold)
\label{sec:work_overflow}

\subsection{Insert New Event} % (fold)
\label{sub:insert_new_event}
	Figure~\ref{fig:storyboard-home} shows the 
% subsection insert_new_event (end)

\subsection{Update Event} % (fold)
\label{sub:update_event}


% subsection update_event (end)

\subsection{Delete Event} % (fold)
\label{sub:delete_event}

	Delete event is very simple in this app. In food list tab, you'll see a list of 
% subsection delete_event (end)

\subsection{Searching Events} % (fold)
\label{sub:searching_logic}
	
	Searching events is done with MapKit API. All the events are fetched in during loading time. The events are annotated with red pins when the tab is loaded.   \\
	As shown in Figure~\ref{fig:map-address}, as the user is typing in the address, the searching string is passed to search controller, so at the same time, we're using geo coder to guess the possible locations for that string. The possible locations are displayed in a table view. After the use choose one possible location, the central region will be focused on that area. At this time, the nearby events stored in database will appear in front of the user.
	

% subsection searching_logic (end)
\newpage
% section work_overflow (end)
\section{User Interface Design} % (fold)
\label{sec:user_interface_design}

\subsection{Home Tab} % (fold)
\label{sub:tabs}
	As shown in Figure~\ref{fig:hometab}, home tab is created for adding new event to the app. The default view for user entering the app is Figure~\ref{fig:guideview}. \\
	
	Figure~\ref{fig:guideview} gives a short tutorial for user. If the user chooses the plus sign in navigation bar, a another blank view should show up. After tapping on the camera icon, Figure~\ref{fig:pickaction} will display an action list including ``Use Last Photo Taken'', ``Take Photo'', ``Choose from Library'' and ``Cancel''. For instance, we choose ``Take Photo'', the app should pop up a modal and let user take picture of food. A modal like Figure~\ref{fig:moveandscale} should appear for user to move and scale the photo to a right position and size. Figure~\ref{fig:didpick} will show up after scaling step. If no photo is chosen, a warning will show up to alert user add a photo first. Tap on next to move to next view Figure~\ref{fig:moreinfo}. In this view, a form is created for user to fill in location information, date, tags, comment and rate. Since the user just starts to eat, comment and 
rate field can be empty for now. Other fields should be filled correctly in this view because the user won't be able to edit all the other fields. In Figure~\ref{fig:locationpicker}, the location list is generated through Foursquare API. Foursquare API is chosen here because it has a good reputation in both academic and industry world. By passing current longitude and latitude, we're able to have a venue list based the distance.

\begin{figure}
    \centering
    \SetFigLayout{3}{3}
    \subfigure[Guide View]{
	\label{fig:guideview}
	\includegraphics[width=\figwidth, totalheight=\figheight, keepaspectratio]{./screenshots/home.png}} \hfill
    \subfigure[Pick Action]{
	\label{fig:pickaction}
	\includegraphics[width=\figwidth, totalheight=\figheight, keepaspectratio]{./screenshots/home-pickaction.png}} \hfill
	\subfigure[Move and Scale]{
	\label{fig:moveandscale}
	\includegraphics[width=\figwidth, totalheight=\figheight, keepaspectratio]{./screenshots/home-moveandscale.png}} \hfill \\
    \subfigure[After Picking]{
	\label{fig:didpick}
	\includegraphics[width=\figwidth, totalheight=\figheight, keepaspectratio]{./screenshots/home-didpick.png}} \hfill
	\subfigure[Event Form]{
	\label{fig:moreinfo}
	\includegraphics[width=\figwidth, totalheight=\figheight, keepaspectratio]{./screenshots/home-moreinfocontd.png}}   \hfill
    \subfigure[Date Picker]{
	\label{fig:datepicker}
	\includegraphics[width=\figwidth, totalheight=\figheight, keepaspectratio]{./screenshots/home-when.png}} \hfill \\
    \subfigure[Location Picker]{
	\label{fig:locationpicker}
	\includegraphics[width=\figwidth, totalheight=\figheight, keepaspectratio]{./screenshots/home-where.png}} \hfill
	\subfigure[Comment Editor]{
	\label{fig:commenteditor}
	\includegraphics[width=\figwidth, totalheight=\figheight, keepaspectratio]{./screenshots/home-comment.png}}  \hfill
	\subfigure[Rate Picker]{
	\label{fig:ratepicker}
	\includegraphics[width=\figwidth, totalheight=\figheight, keepaspectratio]{./screenshots/home-rate.png}} \hfill
	\caption{Home Tab View}
	\label{fig:hometab}
\end{figure}



% subsection tabs (end)

\subsection{Food List Tab} % (fold)
\label{sub:foodie_list_tab}

   In food list tab, it fetches a list of food events ordered by creation date. Figure~\ref{fig:foodlist} shows some events. On each table cell, it shows a thumbnail, time string and string of the event. And a green menu pop up button is created for sharing with social media(Figure~\ref{fig:foodlist-share}) and updating comment(Figure~\ref{fig:foodlist-comment}) or rate(Figure~\ref{fig:foodlist-rate}). \\
   
   After choosing one event, a detail view of the event will appear as shown in Figure~\ref{fig:foodlist-detail} and Figure~\ref{fig:foodlist-detailcontd}. You could see all the detailed information when you create the event. In the top navigation bar, a share button is displayed for user to share the event with friends. If the user chooses twitter, it'll check if he is logged in or not. After making sure the user have a twitter account, Figure~\ref{fig:foodlist-twitter} will appear. The comment and photo is generated by the app. Simply tap on send, and the event will be shared. \\
   

\begin{figure}
    \centering
    \SetFigLayout{3}{3}
    \subfigure[Food List]{
	\label{fig:foodlist}
	\includegraphics[width=\figwidth, totalheight=\figheight, keepaspectratio]{./screenshots/foodlist.png}} \hfill
	\subfigure[Menu Popup]{
	\label{fig:foodlist-menu}
	\includegraphics[width=\figwidth, totalheight=\figheight, keepaspectratio]{./screenshots/foodlist-menupop.png}} \hfill 
	\subfigure[Comment Editor]{
	\label{fig:foodlist-comment}
	\includegraphics[width=\figwidth, totalheight=\figheight, keepaspectratio]{./screenshots/foodlist-comment.png}} \hfill \\
    \subfigure[Rate Action Sheet]{
	\label{fig:foodlist-rate}
	\includegraphics[width=\figwidth, totalheight=\figheight, keepaspectratio]{./screenshots/foodlist-rate.png}} \hfill
	\subfigure[Sharing Activity Controller]{
	\label{fig:foodlist-share}
	\includegraphics[width=\figwidth, totalheight=\figheight, keepaspectratio]{./screenshots/foodlist-share.png}} \hfill 
	\subfigure[Detail View]{
	\label{fig:foodlist-detail}
	\includegraphics[width=\figwidth, totalheight=\figheight, keepaspectratio]{./screenshots/foodlist-detail.png}} \hfill \\
	\subfigure[Detail View Cont'd]{
	\label{fig:foodlist-detailcontd}
	\includegraphics[width=\figwidth, totalheight=\figheight, keepaspectratio]{./screenshots/foodlist-detailcontd.png}} \hfill
	\subfigure[Twitter]{
	\label{fig:foodlist-twitter}
	\includegraphics[width=\figwidth, totalheight=\figheight, keepaspectratio]{./screenshots/foodlist-twitter.png}} \hfill
	\subfigure[Delete View]{
	\label{fig:foodlist-delete}
	\includegraphics[width=\figwidth, totalheight=\figheight, keepaspectratio]{./screenshots/foodlist-delete.png}} \hfill
	\caption{Foodie List Tab View}
	\label{fig:foodielisttab}
\end{figure}

% subsection foodie_list_tab (end)

\subsection{Stats Tab} % (fold)
\label{sub:stats_tab}

Statistics tab is supposed to give the user an overview of all events he created. As shown in Figure~\ref{fig:stats}, ``Rates'' are clustered in different categories. All the tags are counted and shown in the table. The total number of places are shown in the table as well. In this way, the user can easily get an idea of what kind of places and food he likes.
  
\begin{figure}
	\centering
    \SetFigLayout{1}{1}
    {\includegraphics[%
    width=\figwidth, totalheight=\figheight, keepaspectratio]{./screenshots/stats.png}}
    \caption{Statistics Tab View}
	\label{fig:stats}
\end{figure}

% subsection stats_tab (end)
\subsection{Map Tab} % (fold)
\label{sub:map_tab}

In Figure~\ref{fig:map}, all the event locations are labeled with red pins. By default, it is showing the events around your current location. But you could also change your view by passing address in Figure~\ref{fig:map-address}. After choosing a location in the list, the tableview will be dismissed and the new central region will be the location you chose. 

\begin{figure}
	\centering
    \SetFigLayout{1}{2}
	\subfigure[Map]{
	\label{fig:map}
	\includegraphics[width=\figwidth, totalheight=\figheight, keepaspectratio]{./screenshots/map.png}} \hfill
	\subfigure[Address Searching]{
	\label{fig:map-address}
	\includegraphics[width=\figwidth, totalheight=\figheight, keepaspectratio]{./screenshots/map-address.png}} \hfill
    \caption{Map Tab View}
\end{figure}

% subsection map_tab (end)

\subsection{Setting Tab} % (fold)
\label{sub:setting_tab}

Setting tab provides a view to setup configurations of the app (Figure~\ref{fig:settings}). The view in this tab is a static table view built in Storyboard. A web-view controller is created to show the app website and author introduction. By turning on the save photo option, it will save the photo to a album. The feedback option is used for user to submit feedback through TestFlight. TestFlight is a online tool to do open beta testing on the fly. By hooking with TestFlight, we will be able to see all the crash reports, time durations for each session, and even ask questions to users after a checkpoint is reached.\\


\begin{figure}
	\centering
    \SetFigLayout{2}{2}
	\subfigure[Settings]{\includegraphics[width=\figwidth, totalheight=\figheight, keepaspectratio]{./screenshots/settings.png}} \hfill
	\subfigure[Save to Album]{
	\label{fig:album}
	\includegraphics[width=\figwidth, totalheight=\figheight, keepaspectratio]{./screenshots/settings-album.png}} \hfill \\
	\subfigure[Author Web-view]{
	\label{fig:author}
	\includegraphics[width=\figwidth, totalheight=\figheight, keepaspectratio]{./screenshots/settings-author.png}} \hfill
	\subfigure[Website View]{
	\label{fig:website}
	\includegraphics[width=\figwidth, totalheight=\figheight, keepaspectratio]{./screenshots/settings-website.png}} \hfill
    \caption{Setting Tab View}
	\label{fig:settings}
\end{figure}

% subsection setting_tab (end)
% section user_interface_design (end)
